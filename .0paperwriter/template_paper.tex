% APMCM Competition Paper Template
% Based on 2023-2024 APMCM LaTeX templates

\documentclass[12pt,a4paper]{article}

% ==================== Packages ====================
\usepackage[utf8]{inputenc}
\usepackage[english]{babel}
\usepackage{amsmath,amssymb,amsthm}
\usepackage{graphicx}
\usepackage{booktabs}
\usepackage{algorithm2e}
\usepackage{hyperref}
\usepackage[margin=2.5cm]{geometry}
\usepackage{float}
\usepackage{subcaption}
\usepackage{multirow}
\usepackage{cite}

% ==================== Custom Commands ====================
\newcommand{\R}{\mathbb{R}}
\newcommand{\E}{\mathbb{E}}
\newcommand{\Var}{\text{Var}}
\newcommand{\Cov}{\text{Cov}}

% ==================== Document Info ====================
\title{\textbf{Development Analysis and Forecasting of [Topic]}}
\author{Team \#[Your Team Number]}
\date{\today}

% ==================== Hyperref Setup ====================
\hypersetup{
    colorlinks=true,
    linkcolor=blue,
    citecolor=blue,
    urlcolor=blue
}

\begin{document}

\maketitle

% ==================== Abstract ====================
\begin{abstract}
% Background (1-2 sentences)
[Topic] has become increasingly important in recent years due to [reason]. Understanding [specific aspect] is crucial for [stakeholders] to make informed decisions.

% Problem statement (1 sentence)
This paper addresses the problem of [specific problem statement from competition].

% Approach (2-3 sentences)
We develop a comprehensive analytical framework combining time series analysis, regression modeling, and sensitivity analysis. Our approach involves [key step 1], [key step 2], and [key step 3]. We validate our models using historical data and cross-validation techniques.

% Key results (2-3 sentences)
Our analysis reveals that [key finding 1]. The forecasting results indicate [key finding 2] with an average error of [X]\%. Sensitivity analysis demonstrates that [key finding 3].

% Conclusion (1 sentence)
These findings provide valuable insights for [stakeholders] and suggest that [main conclusion].
\end{abstract}

\noindent\textbf{Keywords:} time series forecasting, regression analysis, sensitivity analysis, [domain-specific term], [another term]

% ==================== Table of Contents ====================
\tableofcontents
\newpage

% ==================== 1. Introduction ====================
\section{Introduction}

\subsection{Background}
In recent years, [topic] has attracted considerable attention from researchers, policymakers, and industry practitioners. According to [source], [relevant statistic or fact]. This trend is driven by several factors including [factor 1], [factor 2], and [factor 3].

The significance of this topic lies in [explain importance]. Understanding [specific aspect] is essential for [stakeholders] to [purpose].

\subsection{Problem Statement}
This paper addresses the following questions posed in the competition:

\begin{enumerate}
    \item \textbf{Question 1:} [Restate question 1 from problem statement]
    \item \textbf{Question 2:} [Restate question 2 from problem statement]
    \item \textbf{Question 3:} [Restate question 3 from problem statement]
    \item \textbf{Question 4:} [Restate question 4 from problem statement]
\end{enumerate}

\subsection{Our Approach}
To address these challenges, we develop a multi-faceted analytical approach:

\begin{itemize}
    \item \textbf{Data Analysis:} We analyze [data description] spanning [time period], including [key variables].

    \item \textbf{Modeling Framework:} We employ ARIMA models for time series forecasting, multiple regression for factor analysis, and optimization techniques for [specific purpose].

    \item \textbf{Validation:} We validate our models using cross-validation, residual analysis, and comparison with baseline models.

    \item \textbf{Sensitivity Analysis:} We conduct comprehensive sensitivity analysis to assess model robustness.
\end{itemize}

\subsection{Paper Organization}
The remainder of this paper is organized as follows. Section 2 presents our problem analysis and data exploration. Section 3 lists our assumptions and justifications. Section 4 develops our mathematical models. Section 5 presents model solutions and results. Section 6 conducts sensitivity analysis. Section 7 discusses strengths and weaknesses of our approach. Section 8 concludes.

% ==================== 2. Problem Analysis ====================
\section{Problem Analysis}

\subsection{Problem Decomposition}
We decompose the overall problem into four sub-problems:

\begin{enumerate}
    \item \textbf{Sub-problem 1:} [Description]
    \begin{itemize}
        \item Input: [What data/information is available]
        \item Output: [What needs to be produced]
        \item Approach: [Brief description of method]
    \end{itemize}

    \item \textbf{Sub-problem 2:} [Description]
    \item \textbf{Sub-problem 3:} [Description]
    \item \textbf{Sub-problem 4:} [Description]
\end{enumerate}

\subsection{Data Analysis}
We analyze the provided data in Attachments 1-3. Table~\ref{tab:data_summary} summarizes the key characteristics.

\begin{table}[htbp]
    \centering
    \caption{Summary of Available Data}
    \label{tab:data_summary}
    \begin{tabular}{llll}
        \toprule
        Dataset & Time Period & Variables & Observations \\
        \midrule
        Attachment 1 & 2019-2023 & [Variables] & [N] \\
        Attachment 2 & 2019-2023 & [Variables] & [N] \\
        Attachment 3 & 2019-2023 & [Variables] & [N] \\
        \bottomrule
    \end{tabular}
\end{table}

Figure~\ref{fig:data_exploration} shows the historical trends. We observe that [key observation 1] and [key observation 2].

\begin{figure}[htbp]
    \centering
    \includegraphics[width=0.8\textwidth]{figures/data_exploration.pdf}
    \caption{Historical trends of key variables from 2019 to 2023.}
    \label{fig:data_exploration}
\end{figure}

\subsection{Key Challenges}
The main challenges in this problem are:
\begin{enumerate}
    \item \textbf{Limited data:} Only 5 years of historical data available
    \item \textbf{Multiple factors:} Need to account for [factor 1], [factor 2], etc.
    \item \textbf{Uncertainty:} Future predictions subject to policy changes
\end{enumerate}

% ==================== 3. Assumptions ====================
\section{Assumptions and Justifications}

We make the following assumptions:

\begin{enumerate}
    \item \textbf{Assumption 1:} [State assumption clearly]

    \textit{Justification:} [Explain why this assumption is reasonable]

    \textit{Validity:} [Discuss when this assumption holds]

    \item \textbf{Assumption 2:} Historical trends continue in the near future

    \textit{Justification:} Time series models assume that patterns observed in historical data persist, which is reasonable for short-term forecasts (3 years).

    \textit{Validity:} This assumption is valid in stable economic conditions but may not hold during major disruptions.

    \item \textbf{Assumption 3:} [State assumption]

    \textit{Justification:} [Explain]

    \textit{Validity:} [Discuss]
\end{enumerate}

% ==================== 4. Model Development ====================
\section{Model Development}

\subsection{Model 1: Time Series Forecasting (ARIMA)}

\subsubsection{Model Formulation}
We formulate the forecasting problem using an ARIMA model. Let $y_t$ denote the value of [variable] at time $t$, where $t = 1, 2, \ldots, T$. The ARIMA$(p,d,q)$ model is defined as:

\begin{equation}
    \phi(B)(1-B)^d y_t = \theta(B)\epsilon_t
    \label{eq:arima}
\end{equation}

where:
\begin{itemize}
    \item $B$ is the backshift operator: $B y_t = y_{t-1}$
    \item $\phi(B) = 1 - \phi_1 B - \phi_2 B^2 - \cdots - \phi_p B^p$ is the AR polynomial
    \item $\theta(B) = 1 + \theta_1 B + \theta_2 B^2 + \cdots + \theta_q B^q$ is the MA polynomial
    \item $d$ is the degree of differencing
    \item $\epsilon_t \sim \mathcal{N}(0, \sigma^2)$ is white noise
\end{itemize}

\subsubsection{Parameter Estimation}
We estimate the parameters $(\phi_1, \ldots, \phi_p, \theta_1, \ldots, \theta_q, \sigma^2)$ using maximum likelihood estimation (MLE). The log-likelihood function is:

\begin{equation}
    \ell(\phi, \theta, \sigma^2) = -\frac{T}{2}\log(2\pi\sigma^2) - \frac{1}{2\sigma^2}\sum_{t=1}^{T}\epsilon_t^2
    \label{eq:likelihood}
\end{equation}

We select the order $(p,d,q)$ using the Akaike Information Criterion (AIC):

\begin{equation}
    \text{AIC} = -2\ell + 2k
    \label{eq:aic}
\end{equation}

where $k = p + q + 1$ is the number of parameters.

\subsubsection{Model Justification}
We choose ARIMA because:
\begin{enumerate}
    \item It is well-suited for time series data with trends
    \item It can capture both short-term and long-term dependencies
    \item It has been widely validated in forecasting applications
\end{enumerate}

\subsection{Model 2: Multiple Regression}

\subsubsection{Model Formulation}
To analyze the factors affecting [dependent variable], we develop a multiple regression model:

\begin{equation}
    y_i = \beta_0 + \beta_1 x_{i1} + \beta_2 x_{i2} + \cdots + \beta_p x_{ip} + \epsilon_i
    \label{eq:regression}
\end{equation}

where:
\begin{itemize}
    \item $y_i$ is the dependent variable for observation $i$
    \item $x_{ij}$ are the independent variables (factors)
    \item $\beta_j$ are the regression coefficients
    \item $\epsilon_i \sim \mathcal{N}(0, \sigma^2)$ is the error term
\end{itemize}

\subsubsection{Parameter Estimation}
We estimate the coefficients using ordinary least squares (OLS):

\begin{equation}
    \hat{\boldsymbol{\beta}} = (\mathbf{X}^T\mathbf{X})^{-1}\mathbf{X}^T\mathbf{y}
    \label{eq:ols}
\end{equation}

where $\mathbf{X}$ is the design matrix and $\mathbf{y}$ is the response vector.

% ==================== 5. Results ====================
\section{Model Solution and Results}

\subsection{Question 1: [Question Title]}

\subsubsection{Model Implementation}
We implement the ARIMA model using Python's \texttt{statsmodels} library. After testing various orders, we select ARIMA$(1,1,1)$ based on minimum AIC.

\subsubsection{Results}
Table~\ref{tab:q1_results} presents the forecasting results for the next three years.

\begin{table}[htbp]
    \centering
    \caption{Forecasting Results for Question 1}
    \label{tab:q1_results}
    \begin{tabular}{lrrr}
        \toprule
        Year & Predicted Value & 95\% CI Lower & 95\% CI Upper \\
        \midrule
        2026 & 1250.5 & 1180.2 & 1320.8 \\
        2027 & 1320.3 & 1235.7 & 1404.9 \\
        2028 & 1395.8 & 1295.4 & 1496.2 \\
        \bottomrule
    \end{tabular}
\end{table}

Figure~\ref{fig:q1_forecast} visualizes the historical data and forecasts.

\begin{figure}[htbp]
    \centering
    \includegraphics[width=0.8\textwidth]{figures/q1_forecast.pdf}
    \caption{Historical data (2019-2023) and forecasts (2026-2028) with 95\% confidence intervals.}
    \label{fig:q1_forecast}
\end{figure}

\subsubsection{Model Validation}
We validate the model using:
\begin{itemize}
    \item \textbf{Residual Analysis:} Residuals are approximately normally distributed with mean near zero
    \item \textbf{Ljung-Box Test:} $p = 0.45 > 0.05$, indicating no significant autocorrelation in residuals
    \item \textbf{RMSE:} Root mean squared error is 45.2, representing 3.6\% of the mean value
\end{itemize}

\subsection{Question 2: [Question Title]}
[Similar structure as Question 1]

% ==================== 6. Sensitivity Analysis ====================
\section{Sensitivity Analysis}

To assess the robustness of our models, we conduct sensitivity analysis on key parameters.

\subsection{Parameter: [Parameter Name]}
We vary [parameter] from [min] to [max] while holding other parameters constant. Figure~\ref{fig:sensitivity_param1} shows the results.

\begin{figure}[htbp]
    \centering
    \includegraphics[width=0.7\textwidth]{figures/sensitivity_analysis.pdf}
    \caption{Sensitivity analysis for [parameter]. The output shows [stable/moderate/high] sensitivity to changes in this parameter.}
    \label{fig:sensitivity_param1}
\end{figure}

Our analysis reveals:
\begin{itemize}
    \item When [parameter] increases by 10\%, the output changes by [X]\%
    \item The model is most sensitive to [parameter Y]
    \item The relationship between [parameter] and output is [linear/nonlinear]
\end{itemize}

% ==================== 7. Strengths and Weaknesses ====================
\section{Strengths and Weaknesses}

\subsection{Strengths}
\begin{enumerate}
    \item \textbf{Comprehensive Framework:} Our approach combines multiple modeling techniques, providing robust and cross-validated results.

    \item \textbf{Statistical Rigor:} All models are validated using standard statistical tests and metrics.

    \item \textbf{Practical Applicability:} Our forecasts include confidence intervals, providing decision-makers with uncertainty quantification.

    \item \textbf{Sensitivity Analysis:} We thoroughly examine model robustness to parameter variations.
\end{enumerate}

\subsection{Weaknesses}
\begin{enumerate}
    \item \textbf{Limited Historical Data:} Only 5 years of data may not capture long-term cycles or rare events.

    \item \textbf{Assumption Dependence:} Our forecasts rely on assumptions that may not hold during major disruptions (e.g., pandemics, policy changes).

    \item \textbf{Linear Relationships:} Some models assume linear relationships, which may oversimplify complex dynamics.

    \item \textbf{External Factors:} We do not explicitly model all external factors (e.g., technological changes, consumer preferences).
\end{enumerate}

\subsection{Potential Improvements}
Future work could:
\begin{itemize}
    \item Incorporate additional data sources
    \item Develop nonlinear models (e.g., neural networks)
    \item Include scenario analysis for policy changes
    \item Extend forecasting horizon with uncertainty quantification
\end{itemize}

% ==================== 8. Conclusions ====================
\section{Conclusions}

In this paper, we developed a comprehensive analytical framework to address [problem statement]. Our main findings are:

\begin{enumerate}
    \item \textbf{Finding 1:} [Key result from Question 1]

    \item \textbf{Finding 2:} [Key result from Question 2]

    \item \textbf{Finding 3:} [Key result from Question 3]

    \item \textbf{Finding 4:} [Key result from Question 4]
\end{enumerate}

Our sensitivity analysis demonstrates that the models are [robust/moderately sensitive] to parameter variations, with [parameter X] having the largest impact.

These findings provide valuable insights for [stakeholders]. Specifically, our results suggest that [main conclusion and recommendation].

Future research could extend this work by [potential extension 1] and [potential extension 2].

% ==================== References ====================
\begin{thebibliography}{99}

\bibitem{ref1}
Author, A. (Year). Title of paper. \textit{Journal Name}, Volume(Issue), pages.

\bibitem{ref2}
Author, B. and Author, C. (Year). Title of book. Publisher.

\bibitem{ref3}
Organization. (Year). Title of report. Retrieved from URL.

\end{thebibliography}

% ==================== Appendices ====================
\appendix

\section{Data Tables}
[Include detailed data tables if needed]

\section{Additional Figures}
[Include supplementary figures]

\section{Code}
The complete code for our analysis is available at [repository URL or included below].

\begin{verbatim}
# Main execution script
import numpy as np
import pandas as pd
from statsmodels.tsa.arima.model import ARIMA

# Load data
df = pd.read_csv('data.csv')

# Fit ARIMA model
model = ARIMA(df['value'], order=(1,1,1))
fitted = model.fit()

# Forecast
forecast = fitted.forecast(steps=3)
print(forecast)
\end{verbatim}

\end{document}
